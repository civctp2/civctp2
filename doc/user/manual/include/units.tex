\chapter{Unit Concepts}

In this chapter we discuss concepts related to units in the game.  We describe
what units are and how to obtain and use them.

\section{Purpose}

From the moment you start the game and use a settler to build your first city
you are using units.  In CtP2 units serve as your eyes and ears, your
weapons and your shields.  Some of the purposes to which units can be put are:

\begin{description}
\item[Reconaissance/exploration] Units can explore the map, revealing sites for
cities, opposing civilizations, and bands of barbarians.  A good
exploratory unit is fast and has a large sight range.  Stealth might also be
useful.
\item[Attack] To attack enemy civilizations or barbarians requires
units.  Most of the various properties of units are important to their
effectiveness in attack --- which ones are the most important depends on the
target and the circumstances of the attack.
\item[Defense] Units are needed to defend a civilization from attack from its
enemies, and from infiltration by unconventional warfare units.
\item[Transport] Both naval and air transport units are available which can
carry ground units more quickly than they can move themselves, and across
oceans to places which they could not otherwise reach.
\end{description}

Run through roles of units: recon, attack, defense, bombard, transport, naval
warfare, unconventional warfare, etc.

\section{Properties}

Discuss conventional unit properties: attack, defense, ranged attack, armor,
movement, build cost, support cost (explain special forces, readiness state),
transport capacity, fuel, stealth, veteran status, etc. Refer to appendices to
detailed stats.

\section{Movement \& Combat}

Explain movement restrictions (land, water, mountians, deep water, air),
armies, combat model and veteran units.

\section{Unconventional Warfare}

Explain every unconventional action in the game, organised by types: slavery,
diplomacy, espionage, happiness, wealth, disease.

\section{Managing Units}

User Interface controls and screens related to units, including common orders
(move, attack, expel, disband, etc).

